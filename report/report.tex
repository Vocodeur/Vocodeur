\documentclass[a4paper]{article}

\usepackage[french]{babel}
\usepackage[T1]{fontenc}
\usepackage[utf8]{inputenc}


\begin{document}

	\title{\textbf{Traitement du signal \\ Projet Vocodeur}}
	\author{Marc Heinrich \\ \'Ecole normale supérieure \and Baptiste Lefebvre \\ \'Ecole normale supérieure}
	\date{24 mai 2013}
	\maketitle


\section{Choix techniques}
Aucun.

\section{Difficultés rencontrées}
Aucune.

\section{\'Eléments non réalisés}
Aucun.

\section{Installation}

Une fois l'archive \texttt{heinrich-lefebvre.tgz} récupérée et les fichiers extraits placez-vous dans le répertoire \texttt{heinrich-lefebvre}. Dans ce répertoire la commande :
\begin{itemize}
	\item \texttt{octave vocoder} : exécute le script Octave sur le fichier \texttt{vocal.wav} du répertoire \texttt{wavefiles}
\end{itemize}
veuillez à bien avoir installé au préalable \texttt{Octave}.
Si vous n'avez pas réussi à récupérer l'archive vous pouvez également récupérer le code source à l'aide de la commande
\begin{itemize}
	\item \texttt{git clone https://github.com/Vocodeur/Vocodeur}
\end{itemize}
veuillez à avoir installé au préalable le système de contrôle de version \texttt{git}.
 


\end{document}

